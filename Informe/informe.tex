\documentclass[12pt,a4paper]{article}
% Packages
\usepackage[utf8]{inputenc}
\usepackage[spanish]{babel}
\usepackage{amsmath, amssymb, amsthm}
\usepackage{graphicx}
\usepackage{hyperref}
\usepackage{geometry}
\usepackage{float}
\usepackage{listings}
\usepackage{xcolor}

% Page layout
\geometry{margin=1in}

% Title and author
\title{Informe de Aprendizaje Automático}
\author{Nombre del Estudiante \\ Universidad \\ Curso de Aprendizaje Automático}
\author{
    Chavez, Mauro \\
    \texttt{a@gmail.com}
    \and
    Lewkowicz, Iván \\
    \texttt{a@gmail.com}
    \and
    Drelewicz, Santiago \\
    \texttt{a@gmail.com}
    \and
    Torrez, Matías \\
    \texttt{matiastorrez157@gmail.com}
    \and
    Culaciati, Dante \\
    \texttt{a@gmail.com}
}




\date{\today}

% Custom settings for code listings
\lstset{
    basicstyle=\ttfamily\small,
    backgroundcolor=\color{gray!10},
    frame=single,
    keywordstyle=\color{blue},
    commentstyle=\color{green!60!black},
    stringstyle=\color{red},
    breaklines=true,
    numbers=left,
    numberstyle=\tiny,
    stepnumber=1,
    numbersep=5pt
}

\begin{document}

% Title page
\maketitle
\newpage
\tableofcontents
\newpage

\section{Ejercicio 1}
\par En este inciso se pide separar los datos en conjuntos de entrenamiento y evaluación, donde no se debe utilizar la libreria \texttt{train\_test\_split} de \texttt{sklearn}.

\par Primero se realizo una exploracion de los datos, donde se observa que el dataset posee $200$ features, todas numericas, y $500$ filas.
Se observa que el dataset no tiene valores nulos y que ademas se trata de un problema desbalanceado, donde el $\%70$ de los datos pertenecen a la clase $1$ y el $\%$ restante pertenece a la clase $0$, por lo que no es necesario realizar un preprocesamiento de los datos. 
Se decide entonces utilizar el $80\%$ de los datos para entrenamiento y el $20\%$ restante para evaluacion.

\par Como la proporción de los datos es desbalanceada, realizamos un \texttt{stratified split} en la separación de los datos, procurando mantener la proporción del dataset original para los datos de entrenamiento y evaluación. 


\section{Ejercicio 2}
% Explique la fuente de los datos, su preprocesamiento y las características principales. Incluya gráficos si es necesario.
\par Para la primera parte de este ejercicio, entrenamos un arból de decisión con altura máxima 3 y estimamos la performance del modelo con K fold cross validation para distintas métricas. 

\begin{table}[H]
\centering
\resizebox{\textwidth}{!}{%
\begin{tabular}{|c|c|c|c|c|c|c|}
\hline
\textbf{Permutación} & \textbf{Accuracy} (training) & \textbf{Accuracy} (validación) & \textbf{AUPRC} (training) & \textbf{AUPRC} (validación) & \textbf{AUC ROC} (training) & \textbf{AUC ROC} (validación) \\
\hline
1 & 0.8125   & 0.6375 & 0.6710 & 0.3226 & 0.8058 & 0.5298 \\
\hline
2 & 0.840625 & 0.5875 & 0.7337 & 0.3337 & 0.8458 & 0.5246 \\
\hline
3 & 0.825    & 0.6875 & 0.6431 & 0.3437 & 0.7513 & 0.5811 \\
\hline
4 & 0.81875  & 0.7    & 0.6573 & 0.3626 & 0.7877 & 0.5938 \\
\hline
5 & 0.84375  & 0.65   & 0.6958 & 0.4144 & 0.8085 & 0.5967 \\
\hline  
\textbf{Promedios} &  0.828125 & 0.6525 & 0.6802 & 0.3554 & 0.7998 & 0.5651 \\
\hline
\textbf{Global} & (NO) &  & (NO) &  & (NO) &  \\
\hline
\end{tabular}
}
\caption{Resultados por permutación y métricas}
\label{tab:resultados-permutaciones}
\end{table}


\section{Ejercicio 3}
Describa los algoritmos y técnicas de aprendizaje automático utilizados. Incluya ecuaciones relevantes y justifique sus elecciones.

\section{Ejercicio 4}
Presente los resultados obtenidos, como métricas de evaluación, gráficos de desempeño, etc.

\section{Ejercicio 5}
Analice los resultados, las limitaciones del modelo y posibles mejoras.

\section{Conclusión}
Resuma los hallazgos principales y las conclusiones del informe.

\section*{Referencias}
Incluya las referencias bibliográficas utilizadas en el informe.

\end{document}