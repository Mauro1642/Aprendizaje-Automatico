\documentclass[12pt,a4paper]{article}

% Packages
\usepackage[utf8]{inputenc}
\usepackage[spanish]{babel}
\usepackage{amsmath, amssymb, amsthm}
\usepackage{graphicx}
\usepackage{hyperref}
\usepackage{geometry}
\usepackage{float}
\usepackage{listings}
\usepackage{xcolor}

% Page layout
\geometry{margin=1in}

% Title and author
\title{Informe de Aprendizaje Automático}
\author{Nombre del Estudiante \\ Universidad \\ Curso de Aprendizaje Automático}
\date{\today}

% Custom settings for code listings
\lstset{
    basicstyle=\ttfamily\small,
    backgroundcolor=\color{gray!10},
    frame=single,
    keywordstyle=\color{blue},
    commentstyle=\color{green!60!black},
    stringstyle=\color{red},
    breaklines=true,
    numbers=left,
    numberstyle=\tiny,
    stepnumber=1,
    numbersep=5pt
}

\begin{document}

% Title page
\maketitle
\tableofcontents
\newpage

\section{Ejercicio 1}
En este insciso se pide separar los datos en conjuntos de entranamiento y evaluacion, donde no se debe utilizar la libreria \texttt{train\_test\_split} de \texttt{sklearn}.

Primero se realizo una exploracion de los datos, donde se observa que el dataset posee $200$ features, todas numericas, y $500$ filas.
 Se observa que el dataset no tiene valores nulos y que ademas se trata de un problema desbalanceado, donde el $\%70$ de los datos pertenecen a la clase $1$ 
 y el $\%$ restante pertenece a la clase $0$, por lo que no es necesario realizar un preprocesamiento de los datos. 
 
 Se decide utilizar el $80\%$ de los datos para entrenamiento 
 y el $20\%$ restante para evaluacion.
\section{Ejercicio 2}
Explique la fuente de los datos, su preprocesamiento y las características principales. Incluya gráficos si es necesario.

\section{Ejercicio 3}
Describa los algoritmos y técnicas de aprendizaje automático utilizados. Incluya ecuaciones relevantes y justifique sus elecciones.

\section{Ejercicio 4}
Presente los resultados obtenidos, como métricas de evaluación, gráficos de desempeño, etc.

\section{Ejercicio 5}
Analice los resultados, las limitaciones del modelo y posibles mejoras.

\section{Conclusión}
Resuma los hallazgos principales y las conclusiones del informe.

\section*{Referencias}
Incluya las referencias bibliográficas utilizadas en el informe.

\end{document}